\documentclass[10pt]{beamer}
\usepackage{amsmath,amssymb,longtable,hhline}
\usepackage{mathrsfs}
\usepackage{xcolor}
\usepackage{listings}
\usepackage{hyperref}
\usepackage{multicol}
\usepackage{anyfontsize}
\usepackage{minted}

\usemintedstyle{tango}
\newcommand{\ltprgsize}{\fontsize{5}{5}\selectfont}
\setminted{fontsize=\ltprgsize,mathescape}

\definecolor{mygreen}{rgb}{0,0.6,0}
\definecolor{mygray}{rgb}{0.5,0.5,0.5}
\definecolor{mymauve}{rgb}{0.58,0,0.82}

\hypersetup{
    bookmarks=true,         % show bookmarks bar?
    unicode=true,           % non-Latin characters in Acrobat’s bookmarks
    pdftoolbar=false,        % show Acrobat’s toolbar?
    pdfmenubar=false,        % show Acrobat’s menu?
    pdffitwindow=false,     % window fit to page when opened
    pdfstartview={FitH},    % fits the width of the page to the window
    pdftitle={Компьютерная алгебра в задачах оптимизации},    % title
    pdfauthor={Evgeny Cherkashin, Seseg Badmatsyrenova},     % author
    pdfsubject={symbolic computations},   % subject of the document
    pdfnewwindow=true,      % links in new PDF window
    colorlinks=true,       % false: boxed links; true: colored links
    linkcolor=red,          % color of internal links (change box color with linkbordercolor)
    citecolor=green,        % color of links to bibliography
    filecolor=magenta,      % color of file links
    urlcolor=blue           % color of external links
}

\lstset{language=Python,
  basicstyle=\footnotesize\ttfamily,        % the size of the fonts that are used for the code
  breakatwhitespace=false,         % sets if automatic breaks should only happen at whitespace
  breaklines=true,                 % sets automatic line breaking
  captionpos=b,                    % sets the caption-position to bottom
  commentstyle=\color{mygreen},    % comment style
  escapeinside={\%*}{*)},          % if you want to add LaTeX within your code
  extendedchars=true,              % lets you use non-ASCII characters; for 8-bits encodings only, does not work with UTF-8
%  frame=single,                    % adds a frame around the code
  keepspaces=true,                 % keeps spaces in text, useful for keeping indentation of code (possibly needs columns=flexible)
  keywordstyle=\color{blue},       % keyword style
%  numbers=left,                    % where to put the line-numbers; possible values are (none, left, right)
  numbersep=5pt,                   % how far the line-numbers are from the code
  numberstyle=\tiny\color{mygray}, % the style that is used for the line-numbers
  rulecolor=\color{black},         % if not set, the frame-color may be changed on line-breaks within not-black text (e.g. comments (green here))
  showspaces=false,                % show spaces everywhere adding particular underscores; it overrides 'showstringspaces'
  showstringspaces=false,          % underline spaces within strings only
  showtabs=false,                  % show tabs within strings adding particular underscores
  stepnumber=2,                    % the step between two line-numbers. If it's 1, each line will be numbered
  stringstyle=\color{mymauve},     % string literal style
  tabsize=2,                       % sets default tabsize to 2 spaces
%  title=\lstname                   % show the filename of files included with \lstinputlisting; also try caption instead of
}
\usepackage{pifont}

\usetheme{Warsaw}
\usecolortheme{crane}
%\useinnertheme{rectangles}
\setbeamertemplate{itemize item}{\scriptsize\hbox{\donotcoloroutermaths\ding{113}}}
\setbeamertemplate{itemize subitem}{\tiny\raise1.5pt\hbox{\donotcoloroutermaths$\blacktriangleright$}}
\setbeamertemplate{itemize subsubitem}{\tiny\raise1.5pt\hbox{\donotcoloroutermaths$\blacktriangleright$}}
\setbeamertemplate{enumerate item}{\insertenumlabel.}
\setbeamertemplate{enumerate subitem}{\insertenumlabel.\insertsubenumlabel}
\setbeamertemplate{enumerate subsubitem}{\insertenumlabel.\insertsubenumlabel.\insertsubsubenumlabel}
\setbeamertemplate{enumerate mini template}{\insertenumlabel}

\beamertemplatenavigationsymbolsempty

\usepackage{iftex,ifxetex}
\ifPDFTeX
  \usepackage[utf8]{inputenc}
  \usepackage[T1]{fontenc}
  \usepackage[russian]{babel}
  \usepackage{lmodern}
  \usefonttheme{serif}
\else
  \ifluatex
    \usepackage{unicode-math}
    \defaultfontfeatures{Ligatures=TeX,Numbers=OldStyle}
    \setmathfont{Latin Modern Math}
    \setsansfont{Linux Biolinum O}
    \setmonofont{Fira Mono}
    \usefonttheme{professionalfonts}
    % \setmathfont[
    %     Ligatures=TeX,
    %     Scale=MatchLowercase,
    %     math-style=upright,
    %     vargreek-shape=unicode
    %     ]{euler.otf}
  \fi
\fi

%\useoutertheme{split}
%\useinnertheme{rounded}
\setbeamertemplate{background canvas}[vertical shading][bottom=white!80!cyan!20,top=cyan!10]
%\setbeamertemplate{sidebar canvas left}[horizontal shading][left=white!40!black,right=black]

\graphicspath{{pics/}}


% --------------------------

\begin{document}
\title[Complex large-scale DES under uncertainties: application for automated research]{Complex large-scale resource-driven discrete-event systems under uncertainty: application for automated research}
\author{
\def\and{, }
Evgeny~Cherkashin\and
Qiumei~Cong\and
Igor~Bychkov\and
Nadezhda~Nagul\and
Artem~Davydov\and
Yue~Wang\and
Shi~Huiyuan}


\maketitle

% ----------------------------------------------------------------
\begin{frame}{Large-scale complex Industrial systems}

  With the invention of computers, integrated circuits and communications, information and network communication technology are closely combined with industrial process.

  Industrial systems (IS) are individual manufacturing and network-computational systems. IS have improved overall operation efficiency and reduced consumption of raw materials.

  \begin{description}
  \item[1980] better production quality thanks to improved management,
  \item[1990] better processes organization (reengineering of business processes),
  \item[2000] collaborative manufacturing and management the process between counterparties.
  \end{description}

Due to the large scale of networked systems, it is difficult to realize the traditional centralized control. The nowadays control structures of large-scale networked industrial processes are \emph{decentralized} and \emph{distributed}.  Decentralized control is simpler in structure and more convenient in implementation, but it does not deal with the physical coupling between subsystems.  Distributed control deals with \emph{coupling relationship} through communication between subsystems, the quality of the communication implies the quality of control.

\end{frame}

\begin{frame}{}

\end{frame}

\begin{frame}{Aim of the research}
\textbf{The aim is} the development of new methods of intelligent control synthesis for resource-driven DES with uncertainties is based on
\begin{enumerate}
\item new declarative means in aspects of: event occurrence, state change, resource consumption/production and acquiring, competitor behavior, \emph{etc}.;
\item devising approaches to visual definition of the model with UML, SysML, BPMN, CMMN;
\item development modular supervisors by means of logical inference;
\item adaptation of existing logical inference approaches to support resource-driven games;
\item devising a software for DES description and its computer simulation.
\item represent near-to-real manufacturing processes as RDDESU;
\item testing the methods and software in manufacturing control.
\end{enumerate}

\textbf{Using automation is due to the complexity and scale of the problem.}

\end{frame}

\begin{frame}{Large-scale complex Industrial systems}

\end{frame}

\begin{frame}{Large-scale complex Industrial systems}

\end{frame}


\end{document}

%%% Local Variables:
%%% mode: latex
%%% TeX-master: "talk-2020-09-03-proposal.tex"
%%% End:
